\documentclass[a4paper,12pt,twoside,openany]{report}
%
% Wzorzec pracy dyplomowej
% J. Starzynski (jstar@iem.pw.edu.pl) na podstawie pracy dyplomowej
% Wersja 0.1 - 8 października 2016
%
\usepackage{polski}
\usepackage{helvet}
\usepackage[T1]{fontenc}
\usepackage{anyfontsize}
\usepackage[utf8]{inputenc}
\usepackage[pdftex]{graphicx}
\usepackage{tabularx}
\usepackage{array}
\usepackage[polish]{babel}
\usepackage{subfigure}
\usepackage{amsfonts}
\usepackage{verbatim}
\usepackage{indentfirst}
\usepackage[pdftex]{hyperref}

% Pakiety oraz kolory pomocnicze do kodu źródłowego.
\usepackage{listings}
\usepackage{color}
\definecolor{dkgreen}{rgb}{0,0.6,0}
\definecolor{gray}{rgb}{0.5,0.5,0.5}
\definecolor{mauve}{rgb}{0.58,0,0.82}
\lstset{frame=tb,
  language=Java,
  aboveskip=3mm,
  belowskip=3mm,
  showstringspaces=false,
  columns=flexible,
  basicstyle={\small\ttfamily},
  numbers=none,
  numberstyle=\tiny\color{gray},
  keywordstyle=\color{blue},
  commentstyle=\color{dkgreen},
  stringstyle=\color{mauve},
  breaklines=true,
  breakatwhitespace=true,
  tabsize=2
}


% rozmaite polecenia pomocnicze
% gdzie rysunki?
\newcommand{\ImgPath}{.}

% oznaczenie rzeczy do zrobienia/poprawienia
\newcommand{\TODO}{\textbf{TODO}}


% wyroznienie slow kluczowych
\newcommand{\tech}{\texttt}

% na oprawe (1.0cm - 0.7cm)*2 = 0.6cm
% na oprawe (1.1cm - 0.7cm)*2 = 0.8cm
%  oddsidemargin lewy margines na nieparzystych stronach
% evensidemargin lewy margines na parzystych stronach
\def\oprawa{1.05cm}
\addtolength{\oddsidemargin}{\oprawa}
\addtolength{\evensidemargin}{-\oprawa}

% table span multirows
\usepackage{multirow}
\usepackage{enumitem}	% enumitem.pdf
\setlist{listparindent=\parindent, parsep=\parskip} % potrzebuje enumitem

%%%%%%%%%%%%%%% Dodatkowe Pakiety %%%%%%%%%%%%%%%%%
\usepackage{prmag2017}   % definiuje komendy opieku,nrindeksu, rodzaj pracy, ...


%%%%%%%%%%%%%%% Strona Tytułowa %%%%%%%%%%%%%%%%%
% To trzeba wypelnic swoimi danymi
\title{Tytuł pracy dyplomowej}

% autor
\author{Jakub Młynarczyk}
\nrindeksu{288226}

\opiekun{dr inż. Łukasz Makowski}
\terminwykonania{1 lutego 2019} % data na oświadczeniu o samodzielności
\rok{2019}


% Podziekowanie - opcjonalne
\podziekowania{
  \noindent
{\Large Podziękowania}
\bigskip

Dziękuję bardzo serdecznie wszystkim pracownikom uczelni, rodzinie oraz pracodawcy za poświęcony czas i energię.

\bigskip

{\raggedleft
Jakub Młynarczyk

}


}

% To sa domyslne wartosci
% - mozna je zmienic, jesli praca jest pisana gdzie indziej niz w ZETiIS
% - mozna je wyrzucic jesli praca jest pisana w ZETiIS
%\miasto{Warszawa}
%\uczelnia{POLITECHNIKA WARSZAWSKA}
%\wydzial{WYDZIAŁ ELEKTRYCZNY}
%\instytut{INSTYTUT ELEKTROTECHNIKI TEORETYCZNEJ\linebreak[1] I~SYSTEMÓW INFORMACYJNO-POMIAROWYCH}
% \zaklad{ZAKŁAD ELEKTROTECHNIKI TEORETYCZNEJ\linebreak[1] I~INFORMATYKI STOSOWANEJ}
%\kierunekstudiow{INFORMATYKA}

% domyslnie praca jest inzynierska, ale po odkomentowaniu ponizszej linii zrobi sie magisterska
\pracamagisterska
%%% koniec od P.W

\opinie{%
  \input{opiniaopiekuna.tex}
  \newpage
  \input{recenzja.tex}
}

\streszczenia{
  \newpage
\begin{center}
\large \bf
Problemy współbieżności w algorytmach węzła sieci czujnikowej na przykładzie modelu w języku Go.
\end{center}

\section*{Streszczenie}
Praca składa się z krótkiego wstępu opisującego oraz uzasadniającego cel pracy.
Rozdział drugi `Wykorzystane technologie` opisuje technologie, narzędzia oraz systemu wykorzystane w celu wykonania pracy.
W kolejnym rozdziale, `Bezprzewodowe sieci czujnikowe` przedstawione są podstawowe zagadnienia związane z bezprzewodowymi sieci czujników, algorytmami i protokołami stosowanymi w celach uzyskania lepszej efektywności procesu wymiany informacji.
Materiał ten zostały następnie wykorzystany w rozdziale `Architektura systemu`, dokumentując zebrane wymagania projektowe oraz opisujący implementację autorskiego systemu środowiska symulacyjnego.
Rozdział `Opracowanie wyników eksperymentów` definiuje zakres testów, prezentuje uzyskane wyniki oraz krótko opisuje uzyskane wartości.
Na tej podstawie powstał rozdział `Podsumowanie`, czyli holistyczny opis uzyskanych wyników, zaobserwowanych zależności w bezprzewodowych sieciach czujnikowych oraz możliwościach dalszego rozwoju projektu. 

\bigskip
{\noindent\bf Słowa kluczowe:} wsn, sieci czujnikowe, golang, LEACH, PEGASIS

\vskip 2cm

\newpage
\begin{center}
\large \bf
Challenges of concurrency in wireless sensor network, based on a model developed in Go.
\end{center}

\section*{Abstract}
This thesis presents a novel way of using a novel algorithm to present complex problems of concurrency in wireless sensor networks. 
In the first chapter briefly presents the objectives and goals of the document, as well as introduced the reader to basic concepts of wireless sensor networks.
The second chapter describes all available tools, technologies and utilities used in the process of writing the thesis.
The third and fourth chapter present the fundamentals of wireless sensor networks and technologies, requirements and implementation details of custom-made simulation environment written in Go programming language.
Following chapter defines a set of tests and sub-tests, as well as presents results of those tests, which enable to compare existing wireless sensor network protocols
and proves the correctness of end-to-end simulator.
Final chapter summarizes all the tests results, discovered dependencies and points some new possibilities of further development of the simulator.

\bigskip
{\noindent\bf Keywords:} wsn, wireless sensor networks, golang, LEACH, PEGASIS

\vfill

}

\begin{document}
\maketitle

%-----------------
% Wstęp
%-----------------
\chapter{Wstęp}

\section{Wprowadzenie}
W ciąg ostatnich lat obserwujemy szybki rozwój technologii informatycznych i teleinfromatycznych w zakresie bezprzewodowych sieci czujnikowych.
Pierwsze implementacje bezprzewodowych czujników wykorzystano w celach zbrojeniowych. Aktualnie, możliwości oferowane przez bezprzewodowe czujniki znajdują zastosowanie w nieskończonej 
ilości aplikacji, zarówno w przemysłe (np. medycynie), jak również w sektorze prywatnym (np. inteligentny dom).

\section{Cel pracy}
Celem pracy jest przedstawienie problemów współbieżności w algorymatch sieci czujnikowej. Praca wykorzystuje ogólnie znane i udokumentowane algorytmy umożliwiające
transmitowanie danych w sieci, w których najważniejszym ograniczeniem jest skończona ilość energii czujnika.

Istotnym elementem pracy jest autorski symulator sieci sensorowej, która stanowi podstawowe narzędzie umożliwiąjące implementację oraz testowanie nowych algorytmów.
Aplikacja umożliwia tworzenie symulacji porównujących efekty zastosowania różnych algorytmów transmisji danych, bez potrzeby wykorzystania fizycznych urządzeń.

\section{Tworzenie projektu}
Koncepcja i projekt systemu symulującego model bezprzedowowej sieci czujnikowej został opracy przez autora pracy.
Pierwszym etapem pracy było określenie wymagań oraz funkcjonalności systemu końcowego. Pierwsze szkice dokumentacji były regularnie konsultowane z promotorem pracy.
Proces ten pozwolił na zebranie szczegółowych wymagań projektowych. W końcowym etapie zbierania wymagań, autor zdecydował się zajął się tworzeniem prototypu systemu symulującego. 
Prace te pozwoliły na wczesne przetestowanie założeń i pomysłów dotyczących dekompozycji problemu, których efekt zaowocował uproszczeniem całkowitej architektury systemu.
Projekt zrealizowano w języku Go, przy wykrzystaniu rozszerzonego systmeu kontroli wersji git.

%-----------------
% Wykorzystane technologie
%-----------------
\chapter{Wykorzystane technologie}

Rozdział ten zawiera opis technologii oraz narzędzi wykorzystanych w pracy dyplomowej.

\section{Język programowania Go}
Go (Golang) to język programistyczny stworzony jako wolne oprogramowanie (open source) na potrzeby firmy Google, Inc. 
Głównymi architektami języka są Robert Griesemer, Rob Pike i Ken Thompson.
Golang umożliwia tworzenie komercyjnego oprogramowania i jest wspierany na wielu sytemach opearcyjnych (Linux, Windows, Mac OS X).

Język Go należy do kategorii języków kompilowanych, z statycznym definiowaniem typu zmiennych. Składnia języka jest zbliżona do C/C++ czy Python.

\subsection{Porównanie Go z Python}

\begin{itemize}
 \item Go jest językiem wspomagającym tworzenie aplikacji wielowątkowych.
 \item Go jest językiem statycznym, co pozwala wyeliminować błedy typu runtime wynikające z typu zmiennej.
 \item Go jest językiem samodokumentującym. Określony odgórnie format komentarzy umożliwia automatyczne tworzenie przejrzystej dokumentacji.
 \item Go jest językiem kompilowalnym, co pozwala na szybsze uruchamianie i egzekucję oprogramowania.
 \item Go wykorzystuje mniej pamięci. Przykład: W Go zmienna typu int32 wymaga 4 bajty pamięci, w Python 24 bajty.
 \item Python umożliwia runtime reflection.
 \item Python posiada większy bazę publiczych bibliotek.
\end{itemize}

\subsection{Porównanie Go z C++}

\begin{itemize}
 \item Go posiada system zarządzania pamięcią (garbage collector).
 \item Go jest językiem wspomagającym tworzenie aplikacji wielowątkowych.
 \item Go jest językiem samodokumentującym. Nie wymaga tworzenia plików typu header.
 \item Go nie jest językiem obiektowym. Zdolność dziedziczenia (inheretance) została zastąpiona osadzaniem (embedding).
 \item Go posiada możliwość użycia (zaimportowania) dowolnej bilbioteki C/C++.
 \item C++ posiada osiągnać szybszą egzekujcę oprogramowaia.
 \item C++ posiada tworzenie kodu niezależnie od typu zmiennej (generics).
 \item C++ nie posiada systemu zarządzania pamięcia (garbage collector), co umożliwia większą kontrolę nad zasobami pamięci (np. w mikrokontrolerach).
\end{itemize}

\section{Serializacja danych Protocol Buffers}

Protocol Buffers (proto, protobuf) to mechanizm serializacji danych stworzony na potrzeby firmy Google, Inc.
Protocol Buffers to mechanizm współpracującym niezależnie od języka oprogramowaia aplikacji czy platformy na którym uruchamiana jest aplikacja.
Technologia ta definiuje strukturę danych (proto schema) za pomocą dedykowanego języka, składającego się z prostych zmiennych (np.: int64, string) 
oraz złożonych komunikatów (message).
Struktura ta przechowywana jest w plikach o rozszerzeniu `.proto`, które są następnie kompilowane do dowolnego z wspieranych języków oprogramowania 
(w przypadku Protocol Buffers w wersji 3, wspierana jest generacja kodu w Java, C++, Python, Java Lite, Ruby, JavaScript, Objective-C, C\# oraz PHP).
Natomiast zserializowane dane zostają zapisane w formacie binarnym (wire format), który umożliwia na uzyskanie wyższego poziomu kompresji danych oraz transmisję danych 
bez potrzeby wykonania dalszego kodowania. 
Wynikiem kompilacji plików `.proto` jest zestaw bibliotek zawierający wygenerowany kod źródłowy, wraz z gotowymi strukturami, funkcjami i metodami niezbędnymi do 
operowania danymi w sposób natywny dla wybranego języka programowania.

\begin{lstlisting}
// example.proto
syntax = "proto3";

// Citizen represents a single citizen of Poland.
message Citizen {
  // The name of a citizen.
  string name = 1;
  // The surname of a citizen.
  string surname = 2;
  // (required) Unique Polish national identification number.
  PESEL pesel = 3;
}

// PESEL represents Polish Universal Electronic System for
// Registration of the Population.
message PESEL {
  // (required) Unique Polish national identification number.
  uint64 number = 1;
  bool active = 2;
}
\end{lstlisting}

\section{System kontroli wersji git}

Git to rozproszony system kontroli wersji stworzony jako wolne oprogramowanie (open source). 
Głównymi architektem narzędzia jest Linus Torvalds. Git to oprogramowanie powszechnie stosowanym w przypadku zarządzania oprogramowaniem.
Narzędzie to umożliwia tworzenie pobocznych gałęzi (branch) niezależych od głównej gałęzi. Funkcjonalność ta pozwala na niezależne wprowadzanie zmian
w kodzie na określonej wersji kontrolej, które mogą następnie zostać wprowadzon ponownie do gałęzi głównej (merge).
Architektura rozproszna git (w przeciwieństwie do zcentralizowanych systemów kontroli wersji) umożliwia programistom na posiadanie lokalnej kopii repozytorium,
której zmiany mogą zostać następnie wprowadzone do gałęźi głównej.

\section{GNUPlot}

TODO

\section{Docker}

Docker to narzędzie stworzone jako wolne oprogramowanie (open source) napisane w języku Go przez firmę Docker, Inc.
Narzędzie to pozwala na tworzenie kontenerów, które izolują aplikację na poziomie systmeu operacyjnego. W przeciwieństwie do maszyn wirtualnych, kontener nie wymaga
wirtualizowania systemu operacyjnego dla każdego z kontenerów. Wszystkie równolegle działające kontenery aplikacji działające na pojedynczym urządzeniu współdzielą
parametry fizyczne maszyny oraz jądro systemu operacyjnego (np. Linux). Izolacja kontenerów widoczna jest na poziomie zależności (dependency) do określonych wersji
bibliotek (libraries), narzędzi (binaries), plików konfiguracyjnych czy parametrów.

\section{Latex}

Latex to narzędzie do formatowania tekstu przy wykorzystaniu znaczników. Praca nad dokumentem przy wykorzystaniu Latex ułatwia redagowanie artykułu, ze względu na
odeseparowanie treści od struktury tekstu.

%-----------------
% Architektura systemu
%-----------------
\chapter{Architektura systemu}

\section{Opis}

System informatyczny stworzony na potrzeby pracy dyplomowej ma za zadanie dostarczenie wyników oraz wykresów niezbędnych do zbadania poniższych zależnożci sprawności energetycznej 
węzłów dla wybranych algorytmów w sieci WSN (komunikacji bezpośredniej, LEACH, PEGASIS).

\section{Komponenty systemu}

Poniższy diagram przedstawia dekompozycję modułów, całkowitą architekturę systemu oraz kierunki interakcji i przepływu informacji.

\subsection{Plik konfiguracyjny}

Plik konfiguracyjny (config file) posiada dane wejściowe pozwalające na zbudowanie środowiska testowego. 
Schemat pliku konfiguracyjnego 
Informacje zawarte w pliku konfiguracyjnym można podzielić na dwie sekcje:

\begin{itemize}
 \item konfiguracja symulatora i sieci
 \item konfiguracja węzła
\end{itemize}

\begin{lstlisting}
syntax = "proto3";

package config;

enum E_Protocol {
  UNSET = 0;
  DIRECT = 1;
  LEACH = 2;
  APTEEN = 3;
  PEGASIS = 4;
}

message Config {
  // Simulation protocol.
  E_Protocol protocol = 1;
  // Number of maximum rounds in simulation. 
  int64 max_rounds = 2;
  // Percentage of cluster heads among all nodes [0, 1].
  double p_cluster_heads = 3;
  // Size of data sent by individual node (in Bytes).
  int64 msg_length = 4;
  // Nodes points to configuration for each node.
  repeated Node nodes = 5;
}

// Node defines a configuration for a single node.
message Node {
  // Unique ID for a node.
  int64 id = 1;
  // Initial value of energy (in Joules). 
  double initial_energy = 2;

  // Location of a node in 2D space.
  Location location = 3;
  // Energy consumption of node operations.
  EnergyCost energy_cost = 4;
  // Time delays introduced by node operations
  TimeDelay time_delay = 5;
}

// Location defines a X, Y coordinates of a node.
message Location {
  double X = 1;
  double Y = 2;
}

// EnergyCost defines energy consumption for common node operations.
message EnergyCost {
  // Energy required to transmit one byte (in Joules).
  double transmit = 1;
  // Energy required to receive one byte (in Joules).
  double receive = 2;
  // Energy required to listen the channel for a second (in Joules).
  double listen = 3;
  // Energy required to process sensor data (in Joules).
  double sensor_data_process = 4;
  // Energy required to wake up MCU (in Joules).
  double wake_up_mcu = 5;
}

// TimeDelay defines time delays for common node operations.
message TimeDelay {
  // Time required to process sensor data (in nanoseconds).
  int64 sensor_data_process = 1;
  // Time required to wake up MCU (in nanoseconds).
  int64 wake_up_mcu = 2;
}
\end{lstlisting}

\subsubsection{Konfiguracja symulatora i sieci}

Konfiguracja symulatora i sieci składa się z pięciu zmiennych w komunikacie Config. Każda z tych wartości może być modyfikowana bez potrzeby
ponownej kompilacji oprogramowania symulacyjnego. Poniższa lista przedstawia oraz opisuje znaczenie poszczególnych parametrów konfiguracyjnych:

\begin{itemize}
 \item protokoł (protocol) - zmienna ta zdefiniowana za pomocą komunikatu typu enum E\_Protocol pozwala symulatorowi wybrać odpowiedni 
       protokół sterujący symulacją. Dokładny opis działania protokołów zostanie przedstawiony w dalszej części pracy.
 \item wartości maksymalnej rund pomiarowych (max\_rounds) - zmienna ta pozwala określić wartość rund pomiarowych aktywnych węzłów w pojedynczej symulacji, 
       po której całkowity przebieg symulacji zostanie zatrzymany.
       Wyznaczenie tej wartości umożliwia użytkownikowi określenie dowolnej granicy, bez potrzeby oczekiwania na zakończenie symulacji (wykorzystanie 
       całkowitej energii dostępnej przez węzły pomiarowe).
 \item proporcji węzłów typu kluster do wszystkich węzłów (p\_cluster\_heads) - zmienna ta pozwala określić stosunek ilości węzłów pomiarowych 
       odpowiedzialnych za pośredniczenie w przesyłaniu danych pomiarowych (klastrów) do ilości wszystkich węzłów w sieci. Parametr ten wykorzystywany jest
       zależnie od protokołu.
 \item długość wiadomości (msg\_length) - zmienna ta określa całkowity rozmiar wiadomości generowanych przez pojedynczy węzeł pomiary.
       Długośc wyrażona jest w bajtach i jest sumą dwóch elementów, danych pomiarowych oraz dodatkowych danych generowanych w procesie enkapsulacji
       (np. adresowanie, preambuły, itd.)
 \item konfiguracja węzłów (nodes) - zmienna ta zdefiniowana za pomocą listy komunikatów typu Node. Symulacja musi składać się przynajmniej z dwóch węzłów.
       Kolejność listy ma znaczenie, gdyż pierwszy węzeł pełni rolę głównego odbiornika danych (sink).
       Dokładniejszy opis parametrów poszeczególnych węzłów przedstawiony został w kolejnym porozdziale `Konfiguracja węzła`. 
\end{itemize}

\subsubsection{Konfiguracja węzła}

Konfiguracja węzła składa się z pięciu zmiennych w komunikacje Node. Każda z tych wartości może być modyfikowana bez potrzeby
ponownej kompilacji oprogramowania symulacyjnego. Poniższa lista przedstawia oraz opisuje znaczenie poszczególnych parametrów konfiguracyjnych:

\begin{itemize}
 \item indeks (id) - zmienna ta określa unikatowy identyfikator węzła. Parametr ten umożliwia identyfikację węzła podczas symulacji.
 \item energia początkowa (initial\_energy) - zmienna ta określa ilość energii (mierzonej w [J]), którą posiada węzeł w momencie rozpoczęcia symulacji.
       W przypadku zdefiniowania zerowej energii początkowej, węzeł nie będzie brał udziału w komunikacji ze względu na brak zasobów energetycznych na
       przeprowadzenia jakiejkolwiek operacji.
 \item pozycja (location) - zmienna ta zdefiniowana za pomocą komunikatu typu Location pozwala symulatorowi na umieszenie węzła w dwuwymiarowej przestrzenii.
       Pozycja węzła wykorzystywana jest do określania odległości między węzłami i aplikowania kosztów energetycznych operacji (np. transmisji danych).
 \item koszt energetyczny (energy\_cost) - zmienna ta zdefiniowana za pomocą komunikatu typu EnergyCost pozwala na wprowadzenie dodatkowych kosztów energetycznych
       operacji dla poszczególnych węzłów. Podstawowy model posiada ogórnie zdefiniowane koszty energetyczne (model przedstawiony w sekcji TODO).
       W przypadku zdefiniowana zmiennej EnergyCost dla węzła, koszt operacji (np. transmisji, odbioru, pomiaru i przetwarzania danych) ulega zmianie.
 \item opóźnienia czasowe (time\_delay) - zmienna ta zdefiniowana za pomocą komunikatu typu TimeDelay pozwala symulatorowi na wprowadzenie dodatkowych parametrów czasowych
       dla operacji wykonywanych przez węzeł. Iloczyn zmiennej (np. czasu przetwarzania danych węzła [ns]) i kosztu energetycznego działania węzła [J/s] 
       reprezentuje dodatkowe parametry symulacji, które umożliwiają tworzenie zaawansowanych i precyzyjnych scenariuszy.
\end{itemize}

\subsection{Moduł główny (Core)}

Moduł główny posiada najważniejsze cechy i funkcjonalności umożliwiające modelowanie systemu symulatora WSN.
Wszystkie zaimplementowane struktury (structs) znajdują się w jednej bibliotece (package) o nazwie `core`.

\subsubsection{Węzeł (Node)}

Węzeł (Node) reprezentuje model węzła, będącego elementem podstawowym w sieci. Struktura ta przechowuje dane konfiguracyjne węzła,
poziom energii czy dane historyczne, pozwalające na monitorowanie pracy oraz tworzenie grafów.

Poniżej przedstawiona została struktura węzła oraz definicje funkcji przynależące do struktury.

\begin{lstlisting}
type Node struct {
  Conf    config.Node
  Ready   bool
  nextHop *Node   // As a default set to Base Station.
  Energy  float64 // Energy level of a node.

  transmitQueue int64
  receiveQueue  int64
  // Statistics and aggregation variables.
  dataSent     int64
  dataReceived int64
}

func (n *Node) Transmit(msg int64, dst *Node) error
func (n *Node) Receive(msg int64, src *Node) error
func (n *Node) Info() string

func (n *Node) distance(dst *Node) float64
func (n *Node) consume(e float64) error
\end{lstlisting}

Poniższa lista przedstawia oraz opisuje znaczenie poszczególnych zmiennych struktury węzła (Node):

\begin{itemize}
 \item konfiguracja (Conf) - zmienna publiczna przechowuje konfigurację węzła w formacie protocol buffer (szczegółowe informacje dostępne w `Konfiguracja węzła`).
 \item gotowość (Ready) - zmienna publiczna przechowuje stan gotowości węzła. Wartość `true` oznacza, że węzeł jest gotowy do nadawania i odbierania informacji.
 \item następny skok (nextHop) - zmienna prywatna przechowuje adres do zmiennej węzła, będącego odbiorcą informacji nadawanych przez węzeł. Wartością domyślna w momencie
       rozpoczęcia symulacji jest adres głównego odbiornika danych (sink).
 \item energia (Energy) - zmienna publiczna przechowuje aktualny stan energetyczny węzła. W przypadku wyczerpania energii, zmienna Ready zostaje ustawiona na `false`.
 \item kolejka nadawania (transmitQueue) - zmienna prywatna przechowuje informację o ilości bajtów gotowych do przekazania do następnego węzła na końcu rundy.
 \item kolejka odbioru (receiveQueue) - zmienna prywatna przechowuje informację o ilości bajtów odebranych przez węzeł na początku rundy.
 \item dane nadane (dataSent) - zmienna prywatna przechowuje sumę bajtów nadanych przez węzeł.
 \item dane odebrane (dataReceived) - zmienna prywatna przechowuje sumę bajtów odebranych przez węzeł.
\end{itemize}

Poniższa lista przedstawia oraz opisuje znaczenie poszczególnych funkcji węzła (Node):

\begin{itemize}
 \item Transmit - funkcja pozwala na transmisję danych do węzła.
 \item Receive - funkcja pozwala na odbieranie danych przez węzeł.
 \item Info - funkcja generuje ciąg znaków w podstawowymi informacjami na temat węzła.
 \item distance - funkcja wyznacza wartość odległości pomiędzy dwoma węzłami.
 \item consume - funkcja obciąża energetycznie węzeł.
\end{itemize}

\subsubsection{Sieć (Network)}

Sieć (Network) reprezentuje model środowiska w którym odbywa się symulacja. Struktura ta przechowuje kontroluje przepływ informacji pomiędzy węzłami,
zbiera i eksportuje informacje z poszczególnych rund.

Poniżej przedstawiona została struktura sieci oraz definicje funkcji przynależące do struktury.

\begin{lstlisting}
type Network struct {
  Protocol    Protocol
  BaseStation *Node
  Nodes       sync.Map

  Round     int64
  MaxRounds int64
  MsgLength int64

  GNUPlotNodes       []string
  GNUPlotTotalEnergy []string

  PlotTotalEnergy   *plot.Plot // An amount of total energy in the network per Round.
  PlotNodes         *plot.Plot // A number of alive nodes in the network per Round.
  NodesAlivePoints  plotter.XYs
  NodesEnergyPoints map[int64]plotter.XYs
}

func (net *Network) AddNode(n *Node) error
func (net *Network) Simulate() error
func (net *Network) CheckNodes() int
func (net *Network) PopulateEnergyPoints()
func (net *Network) PopulateNodesAlivePoints()
\end{lstlisting}

Poniższa lista pzedstawia oraz opisuje znaczenie poszeczególnych zmiennych struktury sieci (Network):

\begin{itemize}
 \item protokół (Protocol) - zmienna publiczna przechowuje obiekt definiujący protokół komunikacji pomiędzy węzłami.
 \item stacja bazowa (BaseStation) - zmienna publiczna przechowuje obiekt węzła głównego (sink).
 \item węzły (Nodes) - zmienna publiczna przechowuje obiekty węzłów w sieci. Implementacja przy wykorzystaniu dziennika (hashmap), 
       który umożliwia operacje zapisu i odczytu w procesach równoległych. Kluczem dziennika jest unikatowy identyfikator węzła.
 \item runda (Round) - zmienna publiczna przechowuje numer aktualnej rundy symulacji.
 \item maksymalna ilość rund (MaxRounds) - zmienna publiczna przechowuje maksymalną ilość rund symulacji. 
       W przypadku osiągnęcia wartości Round równej MaxRounds, symulacja zostanie przerwana.
 \item długość wiadomości (MsgLength) - zmienna publiczna przechowuje informację o całkowitej wielkości wiadomości (mierzonej w bajtach) jaka generowana
       jest podczas rundy przez każdy z węzłów. W skład tej wartości wchodzią dane pomiarowe i dodatkowy nakład informacji powstały w wyniku enkapsulacji.
 \item GNUPlotNodes, GNUPlotTotalEnergy - zmienne publiczne przechowujące parametry rund wykorzystywane do generowania grafów przy użyciu narzędzia GNUPlot.
 \item PlotTotalEnergy, PlotNodes, NodesAlivePoints, NodesEnergyPoints - zmienne publiczne przechowujące parametry rund wykorzystywane do generowania grafiów
       przy użyciu biblioteki plotter.
\end{itemize}

Poniższa lista przedstawia oraz opisuje znaczenie poszczególnych funkcji węzła (Node):

\begin{itemize}
 \item AddNode - funkcja pozwala na dodanie węzła do sieci.
 \item Simulate - funkcja umożliwia rozpoczęcie symulacji.
 \item CheckNodes - funkcja sprawdza ilość sprawnych węzłów w sieci.
 \item PopulateEnergyPoints - funkcja sprawdza oraz przechowuje stany energetyczne węzłów.
 \item PopulateNodesAlivePoints - funkcja sprawdza oraz przechowuje ilość sprawnych węzłów w sieci.
\end{itemize}

\subsubsection{Protokół (Protocol)}

Protokół (Protocol) reprezentuje model protokołu komunikacji między wezłami. 

Poniżej przedstawiona został interfejs protokołu oraz definicje funkcji przynależące do interfejsu.

\begin{lstlisting}
type Protocol interface {
  Setup(net *Network) ([]int64, error)
  SetNodes(int)
  SetClusters(int)
}
\end{lstlisting}

Poniższa lista przedstawia oraz opisuje znaczenie poszczególnych funkcji węzła (Node):

\begin{itemize}
 \item Setup - funkcja konfiguruje węzły w sieci zgodnie z zaimplementowanym protokołem.
 \item SetNodes - funkcja definiuje ilość węzłów w sieci. Wartość ta jest niezbędna do wyznaczania parametrów w protokołach (np. APTEEN, PEGASIS).
 \item SetClusters - funkcja definiuje ilość klastrów w sieci. Wartość ta jest niezbędna do wyznaczania parametrów w protokołach (np. APTEEN, PEGASIS).
\end{itemize}

Poniższa lista przedstawia zaimplementowane protokoły:

\begin{itemize}
 \item Komunikacja bezpośrednia	(Direct Communication) - wszystkie węzły w sieci komunikują się bezpośrednio z węzłem głównym (sink).
 \item APTEEN (APTEEN) - wszystkie węzły w sieci komunikują się zgodnie z topologią ustaloną w procesie konfiguracji APTEEN.
 \item PEGASIS (PEGASIS) - wszystkie węzły w sieci komunikują się zgodnie z topologią ustaloną w procesie konfiguracji PEGASIS.
\end{itemize}

\subsection{Moduł symulatora (Simulator)}

Symulator (Simulator) re

\begin{lstlisting}
type Simulator struct {
  namespace map[string]bool
  config    map[string]*config.Config
  network   map[string]*core.Network

  plotTotalEnergy *plot.Plot // An amount of total energy in the network per round.
  plotNodes       *plot.Plot // A number of alive nodes in the network per round.
}

func Create() (*Simulator, error)

func (s *Simulator) AddScenario(name string, conf *config.Config) error
func (s *Simulator) Run() error
func (s *Simulator) ExportPlots(filepath string) error
func (s *Simulator) ExportGNUPlots(filepath string) error

func createAndPopulateFile(filepath string, data []string) error
func (s *Simulator) create(name string, conf *config.Config) error
func createPlot(title, x, y string) (*plot.Plot, error)
func (s *Simulator) plotter() error
\end{lstlisting}

Poniższa lista pzedstawia oraz opisuje znaczenie poszeczególnych zmiennych struktury symulatora (Simulator):

\begin{itemize}
 \item przestrzeń nazw (namespace) - zmienna prywatna przechowuje nazwy symulacji, które w jednoznaczy sposób identyfikują sięć i konfigurację.
 \item przestrzeń konfiguracji (config) - zmienna prywatna przechowuje obiekt konfiguracji (Config) dla każdej symulacji. Kluczem dziennika jest nazwa symulacji.
 \item przestrzeń sieci (network) - zmienna prywatna przechowuje obiekt sieci (Network) dla każdej symulacji. Kluczem dziennika jest nazwa symulacji.
\end{itemize}

Poniższa lista przedstawia oraz opisuje znaczenie poszczególnych funkcji symulatora (Simulator):

\begin{itemize}
 \item Setup - funkcja konfiguruje węzły w sieci zgodnie z zaimplementowanym protokołem.
 \item SetNodes - funkcja definiuje ilość węzłów w sieci. Wartość ta jest niezbędna do wyznaczania parametrów w protokołach (np. APTEEN, PEGASIS).
 \item SetClusters - funkcja definiuje ilość klastrów w sieci. Wartość ta jest niezbędna do wyznaczania parametrów w protokołach (np. APTEEN, PEGASIS).
\end{itemize}

\section{Obsługa systemu}
 
\subsection{Konfiguracja środowiska i kompilacja}

Poprawna konfiguracja środowiska wymaga instalacji niezbędnych bibliotek i pakietów.

\begin{enumerate}
 \item Instalacja kompilatora i bilbiotek Golang w wersji 1.10.1, lub wyższej.
\end{enumerate}

\subsection{Generowanie konfiguracji}

Wcześniej opisany plik konfiguracyjny jest elementem niezbędnym do uruchomienia symulacji.

TODO


\subsection{Uruchamianie scenariuszy}

Przykład uruchamiania projektu z dwoma plikami konfiguracyjnymi:
\begin{lstlisting}
 $ /go/src/github.com/keadwen/msc_project/msc_project \
     --config_file=example1.proto,example2.proto
\end{lstlisting}

Przykład uruchamiania projektu bez pliku konfiguracyjnego:
\begin{lstlisting}
 $ /go/src/github.com/keadwen/msc_project/msc_project
\end{lstlisting}


\subsection{Generowanie wykresów}

TODO

\subsection{Dodawanie nowych protokołów}

Architektura systemu umożliwia tworzenie nowych protokołów wymiany informacji pomiędzy węzłami.
Poprawna implementacja wymaga modifikacji oprogramowania w kilku miejscach:

\begin{enumerate}
 \item Rozszerzenie definicji enum E\_Protocol w proto/config.proto.
 \item Rozszerzenie mapProtocol w simulator/simulator.go
 \item Stworzenie nowego pliku .go w folderze core. Struktura reprezentująca nowy protokołu musi posiadać zestaw funkcji zgodny z interfejsem Protocol.
\end{enumerate}

Po wykonaniu wszystkich z powyższych kroków, protokół może zostać wykorzystany w symulacji.
Należy pamiętać, że oprogramowanie i protocol buffers będą wymagały rekompilacji. W przeciwnym wypadku zmianny nie będą widoczne.

\begin{thebibliography}{99}
\addcontentsline{toc}{chapter}{Bibliografia}
\bibitem{Stevens}{W. R. Stevens, G. R. Wright, ,,Biblia TCP/IP tom 1'', RM, 
1998.}
\end{thebibliography}

\zakonczenie  % wklejenie recenzji i opinii

\end{document}
%+++ END +++
