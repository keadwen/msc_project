\newpage
\begin{center}
\large \bf
Problemy współbieżności w algorytmach węzła sieci czujnikowej na przykładzie modelu w języku Go.
\end{center}

\section*{Streszczenie}
Praca składa się z krótkiego wstępu jasno i wyczerpująco opisującego oraz uzasadniającego cel pracy.
Rozdział drugi `Wykorzystane technologie` opisuje technologie, narzędzia oraz systemu wykorzystane w celu wykonania pracy.
Rozdział trzeci `Bezprzewodowe sieci czujnikowe` przedstawia podstawowe zagadnienia związane z bezprzewodowymi sieci czujników, algorytmami i protokołami
stosowanymi w celach uzyskania lepszej efektywności procesu wymiany informacji.
Rozdział czwarty `Architektura systemu` opisuje wymagania oraz implementację autorskiego systemu środowiska symulacyjnego.
Rozdział piąty `Opracowanie wyników eksperymentów` definiuje zakres testów, prezentuje uzyskane wyniki oraz krótko opisuje uzyskane wartości.
Ostatni rozdział pracy `Podsumowanie` to holistyczny opis uzyskanych wyników, zaobserwowanych zależności w bezprzewodowych sieciach czujnikowych oraz możliwościach dalszego rozwoju projektu. 

\bigskip
{\noindent\bf Słowa kluczowe:} wsn, sieci czujnikowe, golang, LEACH, PEGASIS

\vskip 2cm

\newpage
\begin{center}
\large \bf
Challenges of concurrency in wireless sensor network, based on a model developed in Go.
\end{center}

\section*{Abstract}
This thesis presents a novel way of using a novel algorithm to present complex problems of concurrency in wireless sensor networks. 
In the first chapter briefly presents presents the objectives and goals of the document.
The second chapter describes all available tools, technologies and utilities used in the process of writing.
The third chapter presents the fundamentals of wireless sensor networks and technologies.
The fourth chapter presents requirements and implementation details of custom-made simulation environment written in Go programming language.
The fifth chapter defines a set of tests and sub-tests, as well as presents results of those tests, which enable to compare existing wireless sensor network protocols
and proves the correctness of end-to-end simulator.
Final chapter summarizes all the tests results, discovered dependencies and points some new possibilities of further development of the simulator.

\bigskip
{\noindent\bf Keywords:} wsn, wireless sensor networks, golang, LEACH, PEGASIS

\vfill