\newpage
\begin{center}
\large \bf
Problemy współbieżności w algorytmach węzła sieci czujnikowej na przykładzie modelu w języku Go.
\end{center}

\section*{Streszczenie}
Praca składa się z krótkiego wstępu opisującego oraz uzasadniającego cel pracy.
Rozdział drugi `Wykorzystane technologie` opisuje technologie, narzędzia oraz systemu wykorzystane w celu wykonania pracy.
W kolejnym rozdziale, `Bezprzewodowe sieci czujnikowe` przedstawione są podstawowe zagadnienia związane z bezprzewodowymi sieci czujników, algorytmami i protokołami stosowanymi w celach uzyskania lepszej efektywności procesu wymiany informacji.
Materiał ten zostały następnie wykorzystany w rozdziale `Architektura systemu`, dokumentując zebrane wymagania projektowe oraz opisujący implementację autorskiego systemu środowiska symulacyjnego.
Rozdział `Opracowanie wyników eksperymentów` definiuje zakres testów, prezentuje uzyskane wyniki oraz krótko opisuje uzyskane wartości.
Na tej podstawie powstał rozdział `Podsumowanie`, czyli holistyczny opis uzyskanych wyników, zaobserwowanych zależności w bezprzewodowych sieciach czujnikowych oraz możliwościach dalszego rozwoju projektu. 

\bigskip
{\noindent\bf Słowa kluczowe:} wsn, sieci czujnikowe, golang, LEACH, PEGASIS

\vskip 2cm

\cleardoublepage
\newpage
\begin{center}
\large \bf
Challenges of concurrency in wireless sensor network, based on a model developed in Go.
\end{center}

\section*{Abstract}
This thesis presents a novel way of using a novel algorithm to present complex problems of concurrency in wireless sensor networks. 
In the first chapter briefly presents the objectives and goals of the document, as well as introduced the reader to basic concepts of wireless sensor networks.
The second chapter describes all available tools, technologies and utilities used in the process of writing the thesis.
The third and fourth chapter present the fundamentals of wireless sensor networks and technologies, requirements and implementation details of custom-made simulation environment written in Go programming language.
Following chapter defines a set of tests and sub-tests, as well as presents results of those tests, which enable to compare existing wireless sensor network protocols
and proves the correctness of end-to-end simulator.
Final chapter summarizes all the tests results, discovered dependencies and points some new possibilities of further development of the simulator.

\bigskip
{\noindent\bf Keywords:} wsn, wireless sensor networks, golang, LEACH, PEGASIS

\vfill
